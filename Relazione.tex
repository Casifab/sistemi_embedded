\documentclass[11pt]{article}
\usepackage[utf8x]{inputenc}
\usepackage[italian]{babel}
\usepackage[T1]{fontenc}
\usepackage{hyperref}
\author{
Fabio Casiraghi 807398 \href{mailto:f.casiraghi@campus.unimib.it}{f.casiraghi@campus.unimib.it}
\and
Tommaso Carboni 808431 \href{mailto:t.carboni@campus.unimib.it} {t.carboni@campus.unimib.it}
\and
Giacomo Elemi 806904 \href{mailto:g.elemi@campus.unimib.it} {g.elemi@campus.unimib.it}
}

\title{Progetto Sistemi Embedded}
\begin{document}
\maketitle
\vspace{2cm}

\section*{Livella con sensore di temperatura}
Il progetto, come da indicazioni, rappresenta una livella con sensore di temperatura, il codice eseguito sul microcontrollore calcola gli angoli rispetto all'inclinazione dello stesso rispetto ai 3 assi e le mostra a display. Sotto queste misure viene visualizzata la temperatura percepita dal termometro.
Il progetto è suddiviso su più file .c con rispettivi header. Abbiamo scelto questo tipo di implementazione per alleggerire il più possibile il codice (data la scarsa quantità di memoria disponibile sul microcontrollore) e rendere più chiaro comprendere il compito svolto da ogni segmento di codice.

\subsection{File main.c}
Il file main contiene la routine principale, che si occupa in primo luogo di settare i principali registri del microcontrollore, successivamente inizializza e attiva il timer 4 e per ultimo eseguo un ciclo infinito in cui vengono, ad ogni iterazione, eseguite le funzioni relative a ciascuna delle flag attive.
\begin{itemize}
\item \textbf{init()} Funzione che si occupa di settare i registri principali, in particolare quelli relativi al watchdog, velocità del ciclo di clock, abilita le interrupt globali
\end{itemize}

\end{document}