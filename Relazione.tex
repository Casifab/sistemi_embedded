\documentclass[11pt]{article}
\usepackage[utf8x]{inputenc}
\usepackage[italian]{babel}
\usepackage[T1]{fontenc}
\usepackage{hyperref}
\author{
Fabio Casiraghi 807398 \href{mailto:f.casiraghi@campus.unimib.it}{f.casiraghi@campus.unimib.it}
\and
Tommaso Carboni 808431 \href{mailto:t.carboni@campus.unimib.it} {t.carboni@campus.unimib.it}
\and
Giacomo Elemi 806904 \href{mailto:g.elemi@campus.unimib.it} {g.elemi@campus.unimib.it}
}

\title{Progetto Sistemi Embedded}
\begin{document}
\maketitle
\vspace{2cm}

\section*{Livella con sensore di temperatura}
Il progetto, come da indicazioni, rappresenta una livella con sensore di temperatura, il codice eseguito sul microcontrollore calcola gli angoli rispetto all'inclinazione dello stesso rispetto ai 3 assi e le mostra a display. Sotto queste misure viene visualizzata la temperatura percepita dal termometro.
Il progetto è suddiviso su più file .c con rispettivi header. Abbiamo scelto questo tipo di implementazione per alleggerire il più possibile il codice (data la scarsa quantità di memoria disponibile sul microcontrollore) e rendere più chiaro comprendere il compito svolto da ogni segmento di codice.

\subsection*{File main.c}
Il file main contiene la routine principale, che si occupa in primo luogo di settare i principali registri del microcontrollore, successivamente inizializza e attiva il timer 4 e per ultimo eseguo un ciclo infinito in cui vengono, ad ogni iterazione, eseguite le funzioni relative a ciascuna delle flag attive.
\begin{itemize}
\item \textbf{init()} Funzione che si occupa di settare i registri principali, in particolare quelli relativi al watchdog, velocità del ciclo di clock, abilita le interrupt globali.
\item \textbf{init\_t4()} Funzione che inzializza il timer 4, settando starting e reload value.
\item \textbf{t4()} Eseguita ogni volta che viene generato il segnale d'interrupt 16, legato all'overflow del timer 4 (ogni 100ms). Utilizza dei contatori per tenere il conto del tempo passato e attivare le flag corrispondenti.
\item \textbf{main()} Esegue tutte le inizializzazioni necessarie e avvia il loop infinito che controlla le varie flag ed esegue le funzioni relative se queste flag sono attive.
\end{itemize}

\subsection*{File smbus.c}
Il file contiene l'interrupt service routine relativa al Smbus, necessarie per l'invio e ricezione di dati lungo il bus.
\begin{itemize}
\item \textbf{SM\_Send()} Procedura per l'invio dei dati lungo l'Smbus. Prende in input il dispositivo a cui inviare i dati (chip\_select), la sorgente dei dati da inviare (*src), la quantità di dati da inviare (len) e la modalità relativa alla trasmissione (mode).
\item \textbf{SM\_Receive()} Procedura per la ricezione dei dati dalla scheda lungo l'Smbus. Prende in input il dispositivo da cui ricevere i dati (chip\_select), la destinazione in cui salvare i dati (*dest) e la quantità di dati da leggere (len).
\item \textbf{SW\_Routine()} ISR dell'Smbus, legata all'interrupt 7, che contiene la routine di switch relativa al registro di stato del bus. Ogni stato esegue le procedure standard consigliate nella tabella 1 del pdf \href{https://www.silabs.com/documents/public/application-notes/an113.pdf} {\underline{Serial Communication with the SMBus}} della Silicon Labs. 
\end{itemize}

\subsection*{File acc.c}
File contenente le procedure relative alla lettura e calcolo della posizione, lungo i 3 assi, recepita dall'accelerometro.
\begin{itemize}
\item \textbf{read\_angles()} Funzione che riceve le posizioni calcolate dall'accelerometro e le salva nei rispettivi buffer.
\item \textbf{med\_angles()} Calcola la media delle posizioni contenute in ciascun buffer.
\item \textbf{compose\_line()} Scrive le medie calcolate nelle posizioni corrette del buffer formante la prima linea visualizzata sul display.
\item \textbf{accMain()} Main relativo all'accelerometro, esegue le tre funzioni descritte precedentemente in sequenza.
\end{itemize}

\subsection*{File pwm.c}
Il file contiene le istruzioni necessario per l'inizializzazione e gestione del PWM relativo alla retroilluminazione dell'LCD.
\begin{itemize}
\item \textbf{timer0()} Inizializza il timer 0 settando valori iniziali e di reload
\item \textbf{interrupt\_timer0()} Funzione collegata all'interrupt 1 (overflow timer 0), gestisce l'onda del duty cicle.
\item \textbf{timer2()} Inizializza il timer 2, prende come parametro il valore da cui far ripartire il timer dopo il reload.
\item \textbf{interrupt_timer2()} Legata all'interrupt 5 (overflow timer 2), esegue ogni 10ms 
\end{itemize}
\end{document}